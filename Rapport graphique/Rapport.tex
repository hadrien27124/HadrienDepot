% Options for packages loaded elsewhere
\PassOptionsToPackage{unicode}{hyperref}
\PassOptionsToPackage{hyphens}{url}
%
\documentclass[
]{article}
\usepackage{amsmath,amssymb}
\usepackage{iftex}
\ifPDFTeX
  \usepackage[T1]{fontenc}
  \usepackage[utf8]{inputenc}
  \usepackage{textcomp} % provide euro and other symbols
\else % if luatex or xetex
  \usepackage{unicode-math} % this also loads fontspec
  \defaultfontfeatures{Scale=MatchLowercase}
  \defaultfontfeatures[\rmfamily]{Ligatures=TeX,Scale=1}
\fi
\usepackage{lmodern}
\ifPDFTeX\else
  % xetex/luatex font selection
\fi
% Use upquote if available, for straight quotes in verbatim environments
\IfFileExists{upquote.sty}{\usepackage{upquote}}{}
\IfFileExists{microtype.sty}{% use microtype if available
  \usepackage[]{microtype}
  \UseMicrotypeSet[protrusion]{basicmath} % disable protrusion for tt fonts
}{}
\makeatletter
\@ifundefined{KOMAClassName}{% if non-KOMA class
  \IfFileExists{parskip.sty}{%
    \usepackage{parskip}
  }{% else
    \setlength{\parindent}{0pt}
    \setlength{\parskip}{6pt plus 2pt minus 1pt}}
}{% if KOMA class
  \KOMAoptions{parskip=half}}
\makeatother
\usepackage{xcolor}
\usepackage[margin=1in]{geometry}
\usepackage{color}
\usepackage{fancyvrb}
\newcommand{\VerbBar}{|}
\newcommand{\VERB}{\Verb[commandchars=\\\{\}]}
\DefineVerbatimEnvironment{Highlighting}{Verbatim}{commandchars=\\\{\}}
% Add ',fontsize=\small' for more characters per line
\usepackage{framed}
\definecolor{shadecolor}{RGB}{248,248,248}
\newenvironment{Shaded}{\begin{snugshade}}{\end{snugshade}}
\newcommand{\AlertTok}[1]{\textcolor[rgb]{0.94,0.16,0.16}{#1}}
\newcommand{\AnnotationTok}[1]{\textcolor[rgb]{0.56,0.35,0.01}{\textbf{\textit{#1}}}}
\newcommand{\AttributeTok}[1]{\textcolor[rgb]{0.13,0.29,0.53}{#1}}
\newcommand{\BaseNTok}[1]{\textcolor[rgb]{0.00,0.00,0.81}{#1}}
\newcommand{\BuiltInTok}[1]{#1}
\newcommand{\CharTok}[1]{\textcolor[rgb]{0.31,0.60,0.02}{#1}}
\newcommand{\CommentTok}[1]{\textcolor[rgb]{0.56,0.35,0.01}{\textit{#1}}}
\newcommand{\CommentVarTok}[1]{\textcolor[rgb]{0.56,0.35,0.01}{\textbf{\textit{#1}}}}
\newcommand{\ConstantTok}[1]{\textcolor[rgb]{0.56,0.35,0.01}{#1}}
\newcommand{\ControlFlowTok}[1]{\textcolor[rgb]{0.13,0.29,0.53}{\textbf{#1}}}
\newcommand{\DataTypeTok}[1]{\textcolor[rgb]{0.13,0.29,0.53}{#1}}
\newcommand{\DecValTok}[1]{\textcolor[rgb]{0.00,0.00,0.81}{#1}}
\newcommand{\DocumentationTok}[1]{\textcolor[rgb]{0.56,0.35,0.01}{\textbf{\textit{#1}}}}
\newcommand{\ErrorTok}[1]{\textcolor[rgb]{0.64,0.00,0.00}{\textbf{#1}}}
\newcommand{\ExtensionTok}[1]{#1}
\newcommand{\FloatTok}[1]{\textcolor[rgb]{0.00,0.00,0.81}{#1}}
\newcommand{\FunctionTok}[1]{\textcolor[rgb]{0.13,0.29,0.53}{\textbf{#1}}}
\newcommand{\ImportTok}[1]{#1}
\newcommand{\InformationTok}[1]{\textcolor[rgb]{0.56,0.35,0.01}{\textbf{\textit{#1}}}}
\newcommand{\KeywordTok}[1]{\textcolor[rgb]{0.13,0.29,0.53}{\textbf{#1}}}
\newcommand{\NormalTok}[1]{#1}
\newcommand{\OperatorTok}[1]{\textcolor[rgb]{0.81,0.36,0.00}{\textbf{#1}}}
\newcommand{\OtherTok}[1]{\textcolor[rgb]{0.56,0.35,0.01}{#1}}
\newcommand{\PreprocessorTok}[1]{\textcolor[rgb]{0.56,0.35,0.01}{\textit{#1}}}
\newcommand{\RegionMarkerTok}[1]{#1}
\newcommand{\SpecialCharTok}[1]{\textcolor[rgb]{0.81,0.36,0.00}{\textbf{#1}}}
\newcommand{\SpecialStringTok}[1]{\textcolor[rgb]{0.31,0.60,0.02}{#1}}
\newcommand{\StringTok}[1]{\textcolor[rgb]{0.31,0.60,0.02}{#1}}
\newcommand{\VariableTok}[1]{\textcolor[rgb]{0.00,0.00,0.00}{#1}}
\newcommand{\VerbatimStringTok}[1]{\textcolor[rgb]{0.31,0.60,0.02}{#1}}
\newcommand{\WarningTok}[1]{\textcolor[rgb]{0.56,0.35,0.01}{\textbf{\textit{#1}}}}
\usepackage{longtable,booktabs,array}
\usepackage{calc} % for calculating minipage widths
% Correct order of tables after \paragraph or \subparagraph
\usepackage{etoolbox}
\makeatletter
\patchcmd\longtable{\par}{\if@noskipsec\mbox{}\fi\par}{}{}
\makeatother
% Allow footnotes in longtable head/foot
\IfFileExists{footnotehyper.sty}{\usepackage{footnotehyper}}{\usepackage{footnote}}
\makesavenoteenv{longtable}
\usepackage{graphicx}
\makeatletter
\def\maxwidth{\ifdim\Gin@nat@width>\linewidth\linewidth\else\Gin@nat@width\fi}
\def\maxheight{\ifdim\Gin@nat@height>\textheight\textheight\else\Gin@nat@height\fi}
\makeatother
% Scale images if necessary, so that they will not overflow the page
% margins by default, and it is still possible to overwrite the defaults
% using explicit options in \includegraphics[width, height, ...]{}
\setkeys{Gin}{width=\maxwidth,height=\maxheight,keepaspectratio}
% Set default figure placement to htbp
\makeatletter
\def\fps@figure{htbp}
\makeatother
\setlength{\emergencystretch}{3em} % prevent overfull lines
\providecommand{\tightlist}{%
  \setlength{\itemsep}{0pt}\setlength{\parskip}{0pt}}
\setcounter{secnumdepth}{-\maxdimen} % remove section numbering
% definitions for citeproc citations
\NewDocumentCommand\citeproctext{}{}
\NewDocumentCommand\citeproc{mm}{%
  \begingroup\def\citeproctext{#2}\cite{#1}\endgroup}
\makeatletter
 % allow citations to break across lines
 \let\@cite@ofmt\@firstofone
 % avoid brackets around text for \cite:
 \def\@biblabel#1{}
 \def\@cite#1#2{{#1\if@tempswa , #2\fi}}
\makeatother
\newlength{\cslhangindent}
\setlength{\cslhangindent}{1.5em}
\newlength{\csllabelwidth}
\setlength{\csllabelwidth}{3em}
\newenvironment{CSLReferences}[2] % #1 hanging-indent, #2 entry-spacing
 {\begin{list}{}{%
  \setlength{\itemindent}{0pt}
  \setlength{\leftmargin}{0pt}
  \setlength{\parsep}{0pt}
  % turn on hanging indent if param 1 is 1
  \ifodd #1
   \setlength{\leftmargin}{\cslhangindent}
   \setlength{\itemindent}{-1\cslhangindent}
  \fi
  % set entry spacing
  \setlength{\itemsep}{#2\baselineskip}}}
 {\end{list}}
\usepackage{calc}
\newcommand{\CSLBlock}[1]{\hfill\break\parbox[t]{\linewidth}{\strut\ignorespaces#1\strut}}
\newcommand{\CSLLeftMargin}[1]{\parbox[t]{\csllabelwidth}{\strut#1\strut}}
\newcommand{\CSLRightInline}[1]{\parbox[t]{\linewidth - \csllabelwidth}{\strut#1\strut}}
\newcommand{\CSLIndent}[1]{\hspace{\cslhangindent}#1}
\ifLuaTeX
  \usepackage{selnolig}  % disable illegal ligatures
\fi
\usepackage{bookmark}
\IfFileExists{xurl.sty}{\usepackage{xurl}}{} % add URL line breaks if available
\urlstyle{same}
\hypersetup{
  pdftitle={Rapport graphiques},
  pdfauthor={SCHMITT Hadrien},
  hidelinks,
  pdfcreator={LaTeX via pandoc}}

\title{Rapport graphiques}
\author{SCHMITT Hadrien}
\date{04 février 2025}

\begin{document}
\maketitle

{
\setcounter{tocdepth}{3}
\tableofcontents
}
\begin{figure}

{\centering \includegraphics[width=0.3\linewidth]{rmarkdown} 

}

\caption{Fig. 1: Logo  Rmarkdown}\label{fig:logo}
\end{figure}

\section{Introduction}\label{introduction}

Bienvenu sous RMarkdown. Vous pouvez ici rédiger du texte avec des
polices en \emph{italique}, en \textbf{gras}, ou \textbf{\emph{les
deux}}. Vous pouvez inserer des citations comme ceci (Wynes, Nicholas,
2017), comme cela (blabla, ex : Stern, Wolske, 2017; et blabla, ex :
Lacroix, 2018) ou encore comme ceci Gerlagh \emph{et al.} (2018).
Veillez à bien gérer votre bibliographie en .bib et vos appels à
citation. Vous pouvez utiliser des liens dans le texte comme ceci
\url{https://isaranet.fr/} ou encore utiliser un lien hypertexte comme
\href{https://isaranet.fr/}{\textbf{cela}}. Vous pouvez faire des listes
à tiroir (en sautant bien une ligne avant de commencer) :

\begin{itemize}
\tightlist
\item
  Liste 1

  \begin{itemize}
  \tightlist
  \item
    Sous-liste 1.1
  \item
    Sous-liste 1.2

    \begin{itemize}
    \tightlist
    \item
      Ainsi de suite
    \end{itemize}
  \end{itemize}
\item
  Liste 2
\end{itemize}

Le sommaire sera géré automatiquement grace au bon usage des \# avant le
titre de la partie (ou \#\# ou \#\#\# suivant le niveau du titre).

\section{Les données}\label{les-donnuxe9es}

Les données proviennent de \ldots{}

\subsection{Imports}\label{imports}

RMarkdown permet également d'intégrer directement des codes et sorties
de R dans le rapport. Pour cela vous devez ouvrir un \emph{chunk} avec
une ligne comme
\texttt{\textasciigrave{}\textasciigrave{}\textasciigrave{}\{r\ imports,\ echo=TRUE,eval=TRUE\}}
puis le fermer avec
\texttt{\textasciigrave{}\textasciigrave{}\textasciigrave{}}.
\textbf{Chaque \emph{chunk}} doit avoir un nom différent !. Vous pouvez
choisir d'afficher le code ou non avec \texttt{echo=} et choisir
d'exécuter ce code avec \texttt{eval=} (en général \texttt{TRUE}).

\begin{Shaded}
\begin{Highlighting}[]
\NormalTok{phos }\OtherTok{\textless{}{-}} \FunctionTok{read.table}\NormalTok{(}\StringTok{"phosphates.csv"}\NormalTok{, }\AttributeTok{header =} \ConstantTok{TRUE}\NormalTok{, }\AttributeTok{sep=}\StringTok{";"}\NormalTok{, }\AttributeTok{row.names =} \DecValTok{1}\NormalTok{)}
\end{Highlighting}
\end{Shaded}

\subsection{Présentation}\label{pruxe9sentation}

Nous disposons de données sur \ldots{} :

\begin{Shaded}
\begin{Highlighting}[]
\FunctionTok{head}\NormalTok{(phos)}
\end{Highlighting}
\end{Shaded}

\begin{verbatim}
##       EBL   EUS  EJR   EMR   ESN   ETG   ETN   ECC
## IBL     0 13500    0 36900   240  5160  1140  3060
## ICA     0 85914    0     0     0    86     0     0
## IFR 13350 27450  450 49950 15150 25800 17400     0
## IDL 13590 39060    0 14850  2700  2070  2970 14760
## IIT   413 19293 2006 29559   708  2006  5074     0
## IJP     0 45430 5390 15820  2450   980     0     0
\end{verbatim}

Vous pouvez aussi intégrer du R dans le texte comme dans la phrase
suivante. Nous disposons de 8 exportateurs et 14 importateurs.

\section{Analyses}\label{analyses}

\subsection{Analyses préliminaires}\label{analyses-pruxe9liminaires}

Vous pouvez insérer dans le début du \emph{chunk} des options pour gérer
les figures comme \texttt{fig.height\ =}, \texttt{fig.width\ =} ,
\texttt{fig.align\ =} ou \texttt{fig.cap=} (pour la légende).

\begin{Shaded}
\begin{Highlighting}[]
\NormalTok{import }\OtherTok{\textless{}{-}} \FunctionTok{apply}\NormalTok{(phos, }\DecValTok{1}\NormalTok{, sum)}
\NormalTok{export }\OtherTok{\textless{}{-}} \FunctionTok{apply}\NormalTok{(phos, }\DecValTok{2}\NormalTok{, sum)}
\FunctionTok{par}\NormalTok{(}\AttributeTok{mfrow =} \FunctionTok{c}\NormalTok{(}\DecValTok{2}\NormalTok{, }\DecValTok{1}\NormalTok{), }\AttributeTok{mar =} \FunctionTok{c}\NormalTok{(}\DecValTok{4}\NormalTok{, }\DecValTok{4}\NormalTok{, }\DecValTok{2}\NormalTok{, }\DecValTok{2}\NormalTok{))}
\FunctionTok{barplot}\NormalTok{(}\FunctionTok{sort}\NormalTok{(import),}
  \AttributeTok{horiz =} \ConstantTok{TRUE}\NormalTok{,}
  \AttributeTok{las =} \DecValTok{2}\NormalTok{,}
  \AttributeTok{col =} \StringTok{"slateblue"}\NormalTok{)}
\FunctionTok{barplot}\NormalTok{(}\FunctionTok{sort}\NormalTok{(export),}
        \AttributeTok{horiz =} \ConstantTok{TRUE}\NormalTok{,}
        \AttributeTok{las =} \DecValTok{2}\NormalTok{,}
        \AttributeTok{col =} \StringTok{"red3"}\NormalTok{)}
\end{Highlighting}
\end{Shaded}

\begin{figure}

{\centering \includegraphics{Rapport_files/figure-latex/graph1-1} 

}

\caption{Fig 2: Barplot}\label{fig:graph1}
\end{figure}

Bla bla bla

\subsection{Analyses multivariées}\label{analyses-multivariuxe9es}

\subsubsection{Choix des axes}\label{choix-des-axes}

\begin{Shaded}
\begin{Highlighting}[]
\NormalTok{caphos }\OtherTok{\textless{}{-}} \FunctionTok{CA}\NormalTok{(phos, }\AttributeTok{graph=}\ConstantTok{FALSE}\NormalTok{)}
\DocumentationTok{\#\# Valeurs propres}
\FunctionTok{barplot}\NormalTok{(caphos}\SpecialCharTok{$}\NormalTok{eig[,}\DecValTok{2}\NormalTok{], }\AttributeTok{names.arg =} \FunctionTok{paste}\NormalTok{(}\StringTok{"Fac"}\NormalTok{,}\DecValTok{1}\SpecialCharTok{:}\DecValTok{7}\NormalTok{), }\AttributeTok{las=}\DecValTok{2}\NormalTok{ )}
\end{Highlighting}
\end{Shaded}

\begin{figure}

{\centering \includegraphics{Rapport_files/figure-latex/afc1-1} 

}

\caption{Fig 3 : Intertie}\label{fig:afc1}
\end{figure}

L'axe 1 explique 40.42\% de l'inertie et l'axe 2 en explique 28.4\%.

\subsubsection{Axes et Graphiques}\label{axes-et-graphiques}

\begin{Shaded}
\begin{Highlighting}[]
\FunctionTok{plot.CA}\NormalTok{(caphos)}
\end{Highlighting}
\end{Shaded}

\begin{figure}

{\centering \includegraphics{Rapport_files/figure-latex/afc2-1} 

}

\caption{Fig 4 : Axes}\label{fig:afc2}
\end{figure}

\begin{Shaded}
\begin{Highlighting}[]
\NormalTok{ctrc }\OtherTok{\textless{}{-}}\NormalTok{ caphos}\SpecialCharTok{$}\NormalTok{col}\SpecialCharTok{$}\NormalTok{contrib}
\NormalTok{ctrl }\OtherTok{\textless{}{-}}\NormalTok{ caphos}\SpecialCharTok{$}\NormalTok{row}\SpecialCharTok{$}\NormalTok{contrib}
\end{Highlighting}
\end{Shaded}

\subsection{Classification}\label{classification}

\begin{Shaded}
\begin{Highlighting}[]
\NormalTok{cahp }\OtherTok{\textless{}{-}} \FunctionTok{HCPC}\NormalTok{(caphos, }\AttributeTok{graph=}\ConstantTok{FALSE}\NormalTok{)}
\FunctionTok{plot.HCPC}\NormalTok{(cahp, }\AttributeTok{choice=}\StringTok{"tree"}\NormalTok{)}
\end{Highlighting}
\end{Shaded}

\begin{figure}

{\centering \includegraphics{Rapport_files/figure-latex/arbre-1} 

}

\caption{Fig 5: Arbre}\label{fig:arbre}
\end{figure}

Commentaire :

\begin{Shaded}
\begin{Highlighting}[]
\FunctionTok{lapply}\NormalTok{(cahp}\SpecialCharTok{$}\NormalTok{desc.var, }\ControlFlowTok{function}\NormalTok{(x) }\FunctionTok{return}\NormalTok{(x[x[, }\DecValTok{6}\NormalTok{]}\SpecialCharTok{\textgreater{}}\DecValTok{0}\NormalTok{, ]))}
\end{Highlighting}
\end{Shaded}

\begin{verbatim}
## $`1`
##      Intern %    glob % Intern freq Glob freq  p.value v.test
## EJR  5.665519  2.722962       15634      27270       0    Inf
## EUS 75.490487 35.558866      208316     356116       0    Inf
## 
## $`2`
##      Intern %    glob % Intern freq Glob freq  p.value v.test
## ETN  5.438796  4.238714       35218      42450       0    Inf
## ETG  9.209415  6.061012       59634      60700       0    Inf
## ESN  4.875427  3.633811       31570      36392       0    Inf
## EMR 43.483035 33.745855      281567     337959       0    Inf
## EBL  4.923301  3.210439       31880      32152       0    Inf
## 
## $`3`
##     Intern %    glob % Intern freq Glob freq     p.value   v.test
## ECC     72.8 10.828342       56784     108444 0.00000000      Inf
## ETN      4.4  4.238714        3432      42450 0.02095465 2.308801
\end{verbatim}

Nous pouvons identifier 3 groupes :

\begin{longtable}[]{@{}lll@{}}
\toprule\noalign{}
Numéro & Nom & Caractéristiques \\
\midrule\noalign{}
\endhead
\bottomrule\noalign{}
\endlastfoot
1 & Nom1 & Bla Bla Bla Bla Bla Bla Bla Bla Bla Bla Bla \\
2 & Nom2 & Bla \\
3 & Nom3 & Bla \\
\end{longtable}

\section{Interpretations et
Conclusion}\label{interpretations-et-conclusion}

A vous de jouer\ldots{}

\section{Références}\label{ruxe9fuxe9rences}

\subsection{Liens}\label{liens}

ISARA : \url{https://isaranet.fr/} consulté le \ldots{}

\subsection{Bibliographie}\label{bibliographie}

Liste des publications :

\phantomsection\label{refs}
\begin{CSLReferences}{0}{1}
\bibitem[\citeproctext]{ref-gerlagh2018family}
GERLAGH, Reyer, LUPI, Veronica et GALEOTTI, Marzio, 2018. Family
Planning and Climate Change. 2018.

\bibitem[\citeproctext]{ref-lacroix2018comparing}
LACROIX, Karine, 2018. Comparing the relative mitigation potential of
individual pro-environmental behaviors. \emph{Journal of cleaner
production}. 2018. Vol.~195, pp.~1398‑1407.

\bibitem[\citeproctext]{ref-stern2017limiting}
STERN, Paul C et WOLSKE, Kimberly S, 2017. Limiting climate change:
what's most worth doing? \emph{Environmental Research Letters}. 2017.
Vol.~12, n°~9, pp.~091001.

\bibitem[\citeproctext]{ref-wynes2017climate}
WYNES, Seth et NICHOLAS, Kimberly A, 2017. The climate mitigation gap:
education and government recommendations miss the most effective
individual actions. \emph{Environmental Research Letters}. 2017.
Vol.~12, n°~7, pp.~074024.

\end{CSLReferences}

\end{document}
